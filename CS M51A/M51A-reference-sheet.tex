\documentclass[10pt,letterpaper,twocolumn]{article}

\usepackage[margin=0.3in]{geometry}
\usepackage{amsthm}
\usepackage{amstext}
\usepackage{amsmath}
\usepackage{amssymb}
\usepackage{amsfonts}
\usepackage{ulem}
\usepackage{cancel} % Fancy crossing out
\usepackage[colorlinks=true,linkcolor=red,citecolor=Purple,urlcolor=blue]{hyperref} % Color hyperlinks
\usepackage{graphicx} % Include images
\usepackage{floatrow}
\usepackage[hypcap]{caption} % Jump top of picture
\usepackage{microtype} % Small improvements to type
\usepackage{lettrine} % Drop caps
\usepackage{type1cm}
\usepackage{color} % Colors
\usepackage[usenames,dvipsnames]{xcolor} % More colors
\usepackage{floatrow} % Images side-by-side
\usepackage{lmodern}

\newcommand{\dd}{\mathrm{d}}
\newcommand{\sub}[1]{_{\mathrm{#1}}}
\newcommand{\bb}[1]{\mathbb{#1}}
\newcommand{\force}[1]{\vec{F}_{\mathrm{#1}}}
\newcommand{\sign}[1]{(#1) \,}
\newcommand{\N}{\bb{N}}
\newcommand{\Z}{\bb{Z}}
\newcommand{\Q}{\bb{Q}}
\newcommand{\R}{\bb{R}}
\newcommand{\rr}{\,\textrm{R}\,}
\renewcommand{\text}[1]{\; \mathrm{#1}}
\newcommand{\temp}[2]{#1 \, ^\circ\mathrm{#2}}
\newcommand*\Eval[3]{\left.#1\right\rvert_{#2}^{#3}}
% Binomial coefficient}
\newcommand{\bico}[2]
    {\begin{pmatrix}#1 \\ #2\end{pmatrix}}

\title{Reference Sheet --- CS M51A
                       --- Spring 2015}
\author{Ky-Cuong L. Huynh}
\date{10 June 2015}

\begin{document} 

\maketitle
\frenchspacing

\subsubsection{Specifications and Number Systems}

What (reasonable) assumptions can we make 
about the inputs?
Are there any ``don't care'' input cases?
The bits of the output are
specified separately. Specify the one-set, 
zero-set, and d.c.-set of each $ z $-bit 
separately as well. $ \underbar{z} = z_2z_1z_0 $.
To determine how many bits are needed for $ z $, 
check what the maximum value of the output will be
and remember that $ n $ bits give a max value of
$ 2^n - 1 $. Normal flow: 
(1): Decide I/O and output function. 
(2): Select an encoding scheme. (3): Fill out 
a truth table or a K-map. (4): Minimize and 
get SEs. (5): Implement a network based on SEs.
For multi-level networks: divide-and-conquer, 
factor expressions, bubble logic, 
decompose terms ($ a + b +c + d = (a+b) + (c+d) $
and $ abcd = (ab)(cd) $) to reduce fan-in, and 
buffer to increase output load, and try XOR gates.

For a decimal value $ x $, 
we need at least $ n = \left\lceil \log_{10}(x) 
\right\rceil $ bits. 
Base-$2$ to base-$2^k$: group the
bits together into the same size group
as needed to represent one digit of the new 
base, and then convert and concatenate groups.
Example: \texttt{1110 0110} to
radix-16: \texttt{0xE6}.

To convert from base-$2^k$ to base-$2$, 
we do the opposite with digit expansion. 
Take each radix-8 or radix-16 digit
and convert it into binary, then combine 
them into the final result. Example:
\texttt{0xD7}, then we get \texttt{1101 0111}. 

\emph{Odd parity}: The number of 1's in the 
sequence should be odd. Have the parity bit 
be 0 or 1 to make it so. \texttt{0110110 => 0110101}.
\emph{Even parity}: The number of 1's should be even:
\texttt{001110 => 0011101}.

\subsubsection{Canonical Forms}

SOP (which consists
of \textbf{minterms}) from truth table: 
(1): Identify input rows that give 
1 as output. (2): For each input, write 
either $ a $ for 1 $ a' $ for 0, and 
AND them together. (3) OR (+) this together
with the minterm form of the next 1-row 
$ f $ is true if any of these minterms are true.
Example: $ m3 + m5 $ = $ 011 + 101 $

POS (which consists
of \textbf{maxterms}) from truth table: 
(1): Identify input rows for which
the output is 0 (false). (2) For each input
bit, write $ a $ for 0 or $ a' $ for 1, and
OR the input bits together. (3) AND ($\cdot$)
this maxterm together with the maxterm 
of the next 0-row 
$ f $ is false if any of these maxterms are false
Example: $ M3 \cdot M5 $ = $ 011 + 101 $
= $ (A + B' + C') \cdot (A' + B + C') $
= $ \Pi M(3, 5)$

Minterms to maxterms: Use the maxterms
with complementary indices
Example: $ f(A, B, C) = \sum m(1, 3, 5, 6, 7) 
= \Pi M(0, 2, 4)$.

For maxterms to minterms: 
$ f(A, B, C) = \Pi M(0, 2, 4) 
= \sum m(1, 3, 5, 6, 7) $.

From $ f $ to $ f' $, 
use minterms whose indices do not appear: 
$ f(A, B, C) = \sum m(1, 3, 5, 6, 7) 
\implies f'(A, B, C) = \sum m(0, 2, 4) $. 

From $ f $ to $ f' $ for maxterms, 
$ f(A, B, C) = \Pi M(0, 2, 4) 
\implies f'(A, B, C) = \Pi M(1, 3, 5, 6, 7) $.


\subsubsection{Quine-McCluskey}

1. Implicant table: List out 
implicants (binary form of one-set 1-cells) 
in first column, and group them by number 
of 1's. Pair up implicants that differ only 
by one bit, and check them off as non-primes.
Keep the pair numbers to track which used.

\begin{figure}[H]
    \includegraphics[scale=0.30]
        {./Figures/implicant-table.pdf}
\end{figure}

2. Prime implicant chart: List out the 
prime implicants along the rows (with their 
associated minterm pairs), and the minterms
along the columns. Mark an 'x' for the minterms
that are covered by that prime implicant. 
Circle the essential prime implicants, 
i.e., the ones where a minterm only has that
prime implicant covering it. Draw out 
horizontal lines from the essentials, and 
then vertically from the x's that are 
crossed out. Finally, write a minimal 
switching expression based on the essentials
plus the needed primes (i.e., choose from the 
column entries) to cover any remaining minterms.

\begin{figure}[H]
    \includegraphics[scale=0.30]
        {./Figures/prime-implicant-chart.pdf}
\end{figure}


\subsubsection{Design with PLAs} 

(1): Get a minimal POS expression 
(truth table to K-map to SE). 

(2): Write out a personality matrix
that lists each of the distinct 
product terms, their inputs, and the
outputs they do/do not feed into.
1 if non-negated, 0 if negated, -
if absent. For outputs: 1 if 
product term used, 0 otherwise.

(3): Place a dot to indicate connections, 
going top-down with product terms
first, then OR'ing them together. 
The PLA diagram should follow the structure 
of the personality matrix.

\begin{figure}[H]
    \includegraphics[scale=0.30]{./Figures/PLA.pdf}
\end{figure}

\begin{figure}[H]
    \begin{floatrow}
        \ffigbox{\includegraphics[scale=0.20]
            {./Figures/gate-transform.pdf}}{}
        \ffigbox{\includegraphics[scale=0.30]
            {./Figures/bubble-logic.pdf}}{}
    \end{floatrow}
\end{figure}


\subsubsection{Analysis} 

\emph{Validity} analysis: 
Every gate input connected to 
a signal source, i.e., no floating wires? 
Only one source of signal to 
every input? Any loops, where an 
input into a gate eventually ends up at that
gate again? \emph{Cost}: Network size, power 
dissipation/consumption, monetary cost, etc.

\emph{Functional} analysis: 
1. Get SEs for network outputs in terms of 
network inputs. Divide and conquer
into submodules, then define intermediate variables
for their I/O, and work backwards from outputs
to back-substitute to original inputs. Label
gates starting from inputs, and apply bubble logic 
every other level (starting from output) to simplify 
negation-rich networks. 
2. Get truth table for network outputs. 
3. Define high-level I/O variables and use 
same encoding scheme as the low-level I/O. 
4. Determine the high-level function of network. 


\subsubsection{Timing/Performance Analysis} 

(1):Find the critical path: number of gates (layers), 
delay by gate type (XOR $ > $ AND/OR $ > $ NOT), 
and fan-in (number of inputs) on each gate (more
inputs $ \Rightarrow $ greater delay, as all inputs 
waited for). 

(2):Decide the transition direction (HL or LH) on each
gate, working backward from a starting assumption on
the network output. E.g., if this AND gate gave a 1, 
it underwent a LH transition, and the OR gate feeding 
into it must have had a LH transition.

(3):Label each of the gates, working from the
inputs. Write an expression for the 
total delay in terms of individual gate delays, 
starting from the inputs. 
Example: 
$ T_{pLH}(x_1, z_2) = t_{pLH}(O_1) + t_{pHL}(N_1) $. 
Where $ x_1 $ is an abitrary input that fed 
into the eventual output of $ z_2 $. 

(4): Plug-in propagation delay values from 
the datasheet. Pay attention to the the gate type 
(AND3 vs AND2), transition direction (LH vs. HL) 
and variable loads. We can compute latter by adding 
up load factors on gates it feeds into. 
Clarify on if the question wants this. 
Don't forget units of [ns].


\subsubsection{Signed Arithmetic}

SM of same sign: just add them. 
Of different signs: take one of
the operands and complement it $ + 1 $ to match
signs, then add them, and apply the sign of the 
original operand that was largest. Example: 
$ 10010 + 01101 \Rightarrow 1110 + 1101 = 1 1011 $
And we take the sign of the largest, so $ 01011 $
is our result. 

Get 1's complement: 
Get binary code for $ |x| $, then complement
everything bit-wise. To convert to a decimal 
integer, see if the leftmost is
1. If it's not, treat it as a binary code. 
If it is, it's a negative, and complement
everything before evaluating the magnitude. 
Example: 10011010 is a negative; complement
and we get 01100101 which is 37. 

Get 2's complement: get binary code for $ |x| $, 
then complement bit-wise and add $ +1 $. To convert
to a signed integer, if the leftmost bit is 1, 
then complement everything and add $ +1 $, then 
get the magnitude as a regular binary number. 

Add two 1's comp. numbers: just add them.
If there's a carry-out bit beyond the size of 
our operands, wrap it around
and add it to the initial result. This will happen
once, and then we're done. 2's comp. addition
is the same, but throw away carry-out.

\emph{Overflow} occurs when the last two carry bits
differ, when the MSB suddenly
flips signs from the two operands.
\emph{Sign/range extension} 
is when we copy the MSB repeatedly to get a larger
number of bits that can avoid overflow: 
$ 0100 = 00000100; 1100 = 111111100 $. Before 
carrying out operations, check what the largest 
possible value is, and then extend to match the
number of bits needed to accomodate that.

Evaluate arithmetic expressions: (1): Rewrite
the variables as powers of 2. (2): Range-extend
the result if necessary (based on largest value
produced in expressions). (3) Left-shift to multiply
by 2, right-shift to divide by 2, and complement + 1
to negate (if 2's complement). 

To rewrite something as a power-of-2: (1): Rewrite 
odds as even + odd: $ 9x = (8 + 1)x $. (2): Simplify
terms until powers of 2 exist: 
$ (8 + 1)x/2 = (8x + x)/2 = 4x + x/2 $. (3): Rewrite
powers of 2 explicitly, as those can be accomplished
as left shifts.


\subsubsection{Arithmetic Modules}

\emph{1-bit half adder} (HA) has inputs $ x, y $ 
and outputs $ s, c $ (sum, carry). 1-bit means 
its operands are 1-bit. A \emph{1-bit full adder} (FA)
has inputs $ x, y, c_{\textrm{in}} $ and outputs 
$ s, c_{\textrm{out}} $. A \emph{$ n $-bit full adder}
has inputs: two $ n $-bit operands $ x, y $ and 
carry-in $ c_{\textrm{in}} $, with outputs: 
$ z $ and $ c_{\textrm{out}} $. 
If the operands are 2's comp. for full adder, 
no change, but carry wrap-around 
for 1's comp. (link carry-out to carry-in; loops
at most once).

\begin{figure}[H]
    \includegraphics[scale=0.40]
        {./Figures/ALU.pdf}
\end{figure}


\subsubsection{Sequential System Design} 

\begin{figure}[H]
    \includegraphics[scale=0.30]
        {./Figures/seq-sys-design.pdf}
\end{figure}

If canonical implementation: use D FFs as registers
to store each state bit. Otherwise, use the FF 
excitation tables to figure out what inputs need
to be generated by the combinational network for
next state. Output combinational network will be 
a function of current state (and inputs if Mealy).


\subsubsection{State Minimization}

(1): Get I/O, state, initial state, and 
state transition/output table. (2): Identify
which rows of present states give the same 
outputs, and partition them into state-groups. 
(3): Fill out new state-transition table that
lists which state-group an input-PS combination
will jump to. 
(4): Identify states within groups whose column
differs; they belong in their own group. Single
item groups are automatically good-to-go. Continue
until all columns within groups agree. Now, the states
within groups are equivalent, and the minimal number
of states is equal to the number of groups.

\begin{figure}[H]
    \includegraphics[scale=0.30]
        {./Figures/state-mini.pdf}
\end{figure}


\subsubsection{Sequential System Analysis}

(1): Break loop at input into FFs. (2): Get SEs
for output/next state in terms of PS and external 
inputs. (3): Make a state-transition table based
on all possible PS/inputs based on char. eqns. of FFs.
(4): Define encoding schemes for I/O and states. 
(5): Write a high-level I/O and states spec, plus
a new state-transition/output table. (6): Draw 
a state diagram and/or time-behavior spec (a timeline
of input $ x(t) $ vs. current state $ s(t) $ 
and output $ z(t) $ at each time point).

D (data/delay) FF: basically acts like a register:

\begin{figure}[H]
    \includegraphics[scale=0.30]{./Figures/D-FF.pdf}
    \label{fig:D-FF}
\end{figure}

SR (set/reset) FF: either set current state
to 1, or reset it to 0, but never both (conflicting):

\begin{figure}[H]
    \includegraphics[scale=0.30]{./Figures/SR-FF.pdf}
    \label{fig:SR-FF}
\end{figure}

T (toggle) FF: If 1, then flip the current state, 
else leave it the same: 

\begin{figure}[H]
    \includegraphics[scale=0.30]{./Figures/T-FF.pdf}
    \label{fig:T-FF}
\end{figure}

JK is like SR, but with the 11 condition meaning
''toggle``, and 00 meaning ''hold``. 
It is a universal flip flop; the others can be 
made with it.

\begin{figure}[H]
    \includegraphics[scale=0.30]{./Figures/JK-FF.pdf}
    \label{fig:JK-FF}
\end{figure}


\subsubsection{Combinational Modules}

A \emph{binary decoder} (DEC) takes 
an unsigned integer in binary code of 
$ n $ bits and converts into 
a 1 at the appropriate position among $ 2^n - 1 $
bits. ENABLE it, or else it gives all zeros.
It has NO Active bit output.
It gives the minterms of a function, which 
can then be OR'ed together: $ y_0 + y_5 + y_7 $.
A binary decoder plus OR gates is a universal set.


A \emph{binary encoder} (ENC) takes $ 2^n - 1 $ bits
with a single 1 among them and converts it into 
a unsigned integer in binary code. ENABLE it, 
or else it gives all zeros.
It has an $ A $ bit for ``active'' to 
distinguish between a 1 for the zero-case 
and when all the input bits are truly 0.

\begin{figure}[H]
    \includegraphics[scale=0.30]{./Figures/ENC.pdf}
\end{figure}

To create a code converter with a ENC/DEC pair, 
we take in $ n $ encoded
bits and feed them into a binary decoder
that gives a standardized internal representation
with hot bit. We rewrite these connections into
a binary decoder, which spits out the data
in the other encoding. To figure out the connections,
list the standard ordered binary code as input, 
and figure out what decimal value corresponds 
to that bit pattern in the first encoding.
Then, we link that wire to the corresponding
spot among the encoder's inputs.


A \emph{simple shifter} shifts inputs by one
bit to either the left or the right. 

\begin{figure}[H]
    \includegraphics[scale=0.30]
        {./Figures/shifter-io.pdf}
\end{figure}
A \emph{p-shifter} shifts inputs by $ p $ bits
to the left or to the right. The shift/no-shift
changes to a distance-to-shift input.
A \emph{barrel shifter} shifts inputs in stages, 
with each shifting or not shifting. By wiring
creatively, we shift by different amounts. 


A \emph{multiplexer} (MUX) takes $ 2^n $ inputs
and selects one of them to send on. It combines
multiple data streams for transmission, and sends
one a time, as selected by the $ \underbar{S} $ bits,
where the $ \underbar{S} $ bits are interpreted as 
a unsigned integer. For example, 
$ \underbar{S} = 0110 = 6_{10} $, and 
the input $ x_6 $ would be selected. 

\begin{figure}[H]
    \includegraphics[scale=0.20]{./Figures/MUX.pdf}
\end{figure}

For implementing a switching 
function: (1): List out the truth table. (2): 
If there are more outputs then there are 
inputs available with the MUX, identify which
rows have outputs that are constants (0/1) or 
that can be described in terms of an input bit. 
(3): Feed these in as the MUX's data inputs, and the 
the regular SF's inputs as the selector bits, 
as they choose a row.


A \emph{de-multiplexer} (DEMUX) takes a single
input and sends it out along one of its $ 2^n $ 
outputs, based on the selecting bits. Its diagram
is the reverse of the MUX diagram. Together, a
MUX and DEMUX can be used to combine multiple lines
into one, with them splitting out output among them
and recombining at the end.


\subsubsection{Sequential Modules}

A \emph{register} stores state, and can be
manually cleared asycnchronoously. It updates
only upon clock signal. 

\begin{figure}[H]
    \includegraphics[scale=0.30]{./Figures/REG-io.pdf}
\end{figure}


A \emph{shift register} is a register that supports
shifting operations on current state, inserting 
an input bit into the spot that frees up. 
Four types, each sending in/out 1 or all bits at once.
SI/SO: $ m = n = 1 $. SI/PO: $ m = 1, n > 1 $. 
PI/SO: $ m > 1, n = 1 $. PI/PO: $ m, n > 1 $.
SI/SO has one CTRL bit: shift/no-shift, and one 
right/left insert-bit. SI/PO is the same except
for its parallel outputs. PI/SO has two CTRL bits: 
load/no-load and shift/no-shift. In both the 
serial-out cases, there's a delay of $ n $ cycles
until an inserted bit appears, assuming we shift
(they're unidirectional) each cycle. The PI/PO case: 

\begin{figure}[H]
    \includegraphics[scale=0.30]
        {./Figures/pipo-io.pdf}
\end{figure}

To recognize patterns, 
connect a SI/PO to an AND gate. Each of the expected
0 bits from the output are passed through a NOT gate, 
so that the overall output is not 1 unless 
the previously seen bits were the full pattern. 

A \emph{mod-$p$ counter} counts up to $ p - 1 $, 
and gives $ TC = 1 $ at $ s(t) = p - 1 $ and 
$ x(t) = 1 $.


A \emph{mod-16 counter} has a special LD input
that lets us skip numbers while counting. 
Its native behavior is to increment upon clock
and give a TC output of 1 after a cycle is done.
$ CNT = 1 $ means ``update'' and $ LD = 1 $
means ``use input bits to update''. However, 
if $ CNT = 1 $ and $ LD = 1 $, the count-signal
is ignored, and $ s(t + 1) = \underbar{I} $. 

Have the desired base-state be the input 
$ \underbar{I} $, for $ TC $ and $ LD $, have
an AND gate take in $ x $ and the bits of state 
that would represent the final state before 
wrap-around, e.g., $ s0, s1, s3 $ for 
$ 1011_2 (11_{10})$. Thus, $ TC $ is 1 when 
we wrap-around, and $ LD $ is 1 at the same
time, and so we'll load our base state, which 
might be $ 0001 $, which gets us a 1-to-11 counter. 
If we want modulo-$ k $, then wraparound at $ k - 1 $.
For 1-to-$ k $, wrap-around at $ k $ itself. 
May need combinational networks for CNT, LD, 
inputs, and parallel outputs, with external
input feeding into all of them.

\begin{figure}[H]
    \includegraphics[scale=0.30]
        {./Figures/mod-16.pdf}
\end{figure}



\end{document}

